\documentclass[12pt,a4paper]{article}

\frenchspacing
\sloppy

\usepackage[utf8]{inputenc}
\usepackage{t1enc}
\usepackage[magyar]{babel}
\usepackage{hyperref}
\hypersetup{
    colorlinks=true,
    linkcolor=blue,
    filecolor=magenta,
    urlcolor=blue,
}
\urlstyle{same}
%\usepackage{blindtext}

\title{Önálló laboratórium dokumentáció}
\author{Wágner Árpád (O2OFFX)}
\date{2019/20 - 2. félév}

\begin{document}
  \maketitle

  \section{Bevezetés}
    Ez a dokumentum tartalmazza az önálló laboratórium tárgy keretében elvégzett munkám dokumentációját.
    %TODO: Általános leírás arról, amit csináltunk

    A munka rám háruló része két felé osztható: az első részben feladatom volt egy bizonyos ESP32 lapka megismerése és lehetőségeinek felderítése. A második részben pedig egy, a későbbiekben diagnosztikai eszközként használandó mikrofonnal foglalkoztam.

    \textbf{Megjegyzés:} A laboratórium során létrejött forráskódok \href{https://github.com/awrpad/Onlab}{ezen} a GitHub repositoryn elérhetőek.

  \newpage

  \section{``Nagykijelzős'' ESP32}
    Mint azt a bevezetőben említettem, lorem ipsum...

  \section{Mikrofon és spektrumanalizátor}
    Ahogy...

\end{document}
