\documentclass[12pt,a4paper]{article}

\frenchspacing
\sloppy

\usepackage[utf8]{inputenc}
\usepackage{t1enc}
\usepackage[magyar]{babel}
\usepackage{hyperref}
\usepackage[dvipsnames]{xcolor}
\hypersetup{
    colorlinks=true,
    linkcolor=blue,
    filecolor=magenta,
    urlcolor=blue,
}
\urlstyle{same}
%\usepackage{blindtext}


\title{Önálló laboratórium dokumentáció}
\author{Wágner Árpád (O2OFFX)}
\date{2019/20 - 2. félév}

\begin{document}
  \maketitle

  \section{Bevezetés}

    Ez a dokumentum tartalmazza az önálló laboratórium tárgy keretében elvégzett munkám dokumentációját.
    %TODO: Általános leírás arról, amit csináltunk

    A munka rám háruló része két felé osztható: az első részben feladatom volt egy bizonyos ESP32 lapka megismerése és lehetőségeinek felderítése. A második részben pedig egy, a későbbiekben diagnosztikai eszközként használandó mikrofonnal foglalkoztam.

    \textbf{Megjegyzés:} A laboratórium során létrejött forráskódok \href{https://github.com/awrpad/Onlab}{ezen} a GitHub repository-n elérhetőek.

  \newpage

  \section{``Nagykijelzős'' ESP32}

    \subsection{Bevezetés}
      Mint azt a bevezetőben említettem, a félév első részében egy ESP32-re épülő fejlesztői lapkával foglalkoztam.
      Az ESP32 egy kedvező árú, de erős mikrokontroller-család, melyről érdemes tudni, hogy beépített Wi-Fi-vel és Bluetooth-szal rendelkezik. A szóban forgó eszköz ezen funkcionalitáson túl rendelkezik egy beépített MPU9250-el (giroszkóp, gyorsulásmérő, iránytű), SD kártya olvasóval, egy egyszerű hangszóróval és kijelzővel illetve három előlapi gombbal.

      Az eszköz programzása Arduino IDE-n keresztül lehetséges, de az ESP32 alaplapkönyvtár hozzáadása szükséges (ennak a mikéntjéről például \href{https://randomnerdtutorials.com/installing-the-esp32-board-in-arduino-ide-windows-instructions/}{itt} található egy leírás). Ha ez megtörtént, az \texttt{Eszközök > Alaplap} menüpontnál a felkínált lehetőségek közül válasszuk az \texttt{ESP 32 Pico Kit}-et. Így a megírt programunkat már feltölthetjük az eszközre.

    \subsection{Tapasztalatok}
      \subsubsection{Kijelző}
      A ... könyvtárat érdemes használni...

      \subsubsection{Hangszóró}
      A ... könytár...

      \subsubsection{SD kártya}
      A ... könytár...

    \subsection{Konklúzió}
      Általánosan elmondható, hogy maga a fejlesztői eszköz hasznos és sokrétű. Különféle alkalmazásokban gyakran előforduló elemek találhatóak meg benne beépítve, ami fejlesztés során sok forrasztástól és kábelrengetegről kímélhet meg minket.
      Azonban sajnálatos a megfelelő dokumentáció hiánya, így a vele történő korai munka nehézkes lehet, plusz szerintem a GPIO pinek elhelyezése sem optimális.
      Azonban amint megismerkedtünk a lapkával a rendelkezésre álló mintakódok alapján, összességében egy sokrétűen és jól használható eszköz lesz a kezünkben.

  \section{Mikrofon és spektrumanalizátor}
    Ahogy...

\end{document}
